\documentclass[10pt,twocolumn,oneside]{article}
\setlength{\columnsep}{18pt}                    %兩欄模式的間距
\setlength{\columnseprule}{0pt}                 %兩欄模式間格線粗細

\usepackage{amsthm}                             %定義,例題
\usepackage{amssymb}
\usepackage{fontspec}                           %設定字體
\usepackage{color}
\usepackage[x11names]{xcolor}
\usepackage{listings}                           %顯示code用的
\usepackage{fancyhdr}                           %設定頁首頁尾
\usepackage{graphicx}                           %Graphic
\usepackage{enumerate}
\usepackage{titlesec}
\usepackage{amsmath}
\usepackage[CheckSingle, CJKmath]{xeCJK}
\usepackage{CJKulem}

\usepackage{amsmath, courier, listings, fancyhdr, graphicx}
\topmargin=0pt
\headsep=5pt
\textheight=740pt
\footskip=0pt
\voffset=-50pt
\textwidth=545pt
\marginparsep=0pt
\marginparwidth=0pt
\marginparpush=0pt
\oddsidemargin=0pt
\evensidemargin=0pt
\hoffset=-42pt

%\renewcommand\listfigurename{圖目錄}
%\renewcommand\listtablename{表目錄}

%%%%%%%%%%%%%%%%%%%%%%%%%%%%%

\setmainfont[
    AutoFakeSlant,
    BoldItalicFeatures={FakeSlant},
    UprightFont={* Medium},
    BoldFont={* Bold}
]{Inconsolata}
%\setmonofont{Ubuntu Mono}
\setmonofont[
    AutoFakeSlant,
    BoldItalicFeatures={FakeSlant},
    UprightFont={* Medium},
    BoldFont={* Bold}
]{Inconsolata}
\setCJKmainfont{Noto Sans CJK TC}
\XeTeXlinebreaklocale "zh"                      %中文自動換行
\XeTeXlinebreakskip = 0pt plus 1pt              %設定段落之間的距離
\setcounter{secnumdepth}{3}                     %目錄顯示第三層

%%%%%%%%%%%%%%%%%%%%%%%%%%%%%
\makeatletter
\lst@CCPutMacro\lst@ProcessOther {"2D}{\lst@ttfamily{-{}}{-{}}}
\@empty\z@\@empty
\makeatother
\lstset{                                        % Code顯示
    language=C++,                               % the language of the code
    basicstyle=\footnotesize\ttfamily,          % the size of the fonts that are used for the code
    numbers=left,                               % where to put the line-numbers
    numberstyle=\scriptsize,                    % the size of the fonts that are used for the line-numbers
    stepnumber=1,                               % the step between two line-numbers. If it's 1, each line  will be numbered
    numbersep=5pt,                              % how far the line-numbers are from the code
    backgroundcolor=\color{white},              % choose the background color. You must add \usepackage{color}
    showspaces=false,                           % show spaces adding particular underscores
    showstringspaces=false,                     % underline spaces within strings
    showtabs=false,                             % show tabs within strings adding particular underscores
    frame=false,                                % adds a frame around the code
    tabsize=2,                                  % sets default tabsize to 2 spaces
    captionpos=b,                               % sets the caption-position to bottom
    breaklines=true,                            % sets automatic line breaking
    breakatwhitespace=true,                     % sets if automatic breaks should only happen at whitespace
    escapeinside={\%*}{*)},                     % if you want to add a comment within your code
    morekeywords={*},                           % if you want to add more keywords to the set
    keywordstyle=\bfseries\color{Blue1},
    commentstyle=\itshape\color{Red1},
    stringstyle=\itshape\color{Green4},
}


\begin{document}
\pagestyle{fancy}
\fancyfoot{}
%\fancyfoot[R]{\includegraphics[width=20pt]{ironwood.jpg}}
\fancyhead[C]{基于 K-means 聚类模型的餐饮市场细分与改进策略研究}
\fancyhead[L]{Codebook}
\fancyhead[R]{\thepage}
\renewcommand{\headrulewidth}{0.4pt}
\renewcommand{\contentsname}{Contents}

\scriptsize
\tableofcontents
\section{可视化代码及训练代码}
    \subsection{可视化}
        \lstinputlisting{Contents/section1/visualization.ipynb}
    \subsection{训练}
        \lstinputlisting{Contents/section1/train.ipynb}

\section{K-means 伪代码}
    \documentclass[12pt,a4paper]{article}
\usepackage[UTF8]{ctex}
\usepackage[lined,boxed,commentsnumbered]{algorithm2e}
\begin{document}
\begin{algorithm}[H]
	\label{algo:K-means}
	\SetAlgoLined
    \caption{K-Means}
	\KwIn{Number of nodes $N_A$, Number of blocks $N_B$, Node array $A$}
	\KwOut{Seperating result array R}
	Centers:array[Node], len=$N_B$\;
	Randomly initializing Centers\;
	Cs:Currunt seperation, array[int]\;
	Ls:Last seperation, array[int]\;
	\SetKwFunction{MyFunction}{UpdateCs}
	\SetKwProg{Fn}{Function}{:}{\KwRet}
	\Fn{\MyFunction{}}{
		Ls $\leftarrow$ Cs\;
		\ForEach{$i \in [1,N_A]$}{
			\ForEach{$j \in [1,N_B]$}{
				\If{$dis(A_i,Cs_i)>dis(A_i,Centers_j)$}{$Cs_i$ $\leftarrow$ $Centers_j$}
			}
		}
	}
	\SetKwFunction{MyFunction}{UpdateCenters}
	\SetKwProg{Fn}{Function}{:}{\KwRet}
	\Fn{\MyFunction{}}{
		Csizes:array[int]\;
		\ForEach{$i \in [1,N_B]$}{
			$Csizes_i$ $\leftarrow$ Cs.count(i)\;
			$Centers_i$ $\leftarrow$ 0\;
		}
		\ForEach{$i \in [1,N_A]$}{
			$Centers_{Cs_i}$ $\leftarrow$ $Centers_{Cs_i}$ + $\frac{A_i}{Csizes_{Cs_i}}$\;
		}
	}
	UpdateCs()\;
	\While{$Cs \neq Ls$}{
		UpdateCenters()\;
		UppdateCs()\;
	}
	\Return{$Cs$}\;
\end{algorithm}
\end{document}

\end{document}
